\usetheme{src/sintef}
\usefonttheme[onlymath]{serif}
\titlebackground*{beamerthemesrc/assets/background}
%-------------add your packages here-------------
\usepackage{amsfonts,amsmath,oldgerm}
\usepackage[T1]{fontenc}
\usepackage{tikz, pgfplots}
\usepackage{caption}
\usepackage{subcaption,graphicx}
\usepackage{siunitx}
\usepackage{colortbl}
\usepackage{lipsum}
\usepackage{makecell}
\usepackage{booktabs}
\usepackage{multirow}
\usepackage{mathtools}
\usepackage{appendixnumberbeamer}
%-------------add your commands here-------------
\newcommand{\hrefcol}[2]{\textcolor{cyan}{\href{#1}{#2}}}
\newcommand{\testcolor}[1]{\colorbox{#1}{\textcolor{#1}{test}}~\texttt{#1}}
\pgfplotsset{compat=1.18}
\usetikzlibrary{shapes, arrows, positioning, calc}  
\usetikzlibrary{overlay-beamer-styles}
\usetikzlibrary{positioning}
\usetikzlibrary{shapes.arrows}
\usetikzlibrary{spy}
\usetikzlibrary{patterns}
\usetikzlibrary{backgrounds}
\usetikzlibrary{arrows.meta}
\usetikzlibrary{%
	calc,%
	decorations.pathmorphing,%
	fadings,%
	shadings%
}

% Define graphics path
\graphicspath{{beamerthemesrc/figures/}}

% Semiconductor manufacturing process arrows
\tikzstyle{arrow} = [single arrow, minimum height=1.7cm, single arrow head extend=0.1cm, draw=outlinecolor, fill=outlinecolor, font=\scriptsize, text=textfigurecolor]
% Semiconductor manufacturing process final block
\tikzstyle{finalblock} = [rectangle, minimum height=0.52cm, minimum width=1.7cm, draw=outlinecolor, fill=outlinecolor, font=\scriptsize, text=textfigurecolor]
% Rectangle of the fab process steps
\tikzstyle{step_rectangle} = [rectangle, font=\scriptsize, rounded corners, minimum width=1.2cm, minimum height=0.55cm, text centered, text=textfigurecolor, draw=outlinecolor, fill=outlinecolor]
% Circle for the question mark indicating a novelty
\tikzstyle{novelty} = [circle, font=\scriptsize, text centered, text=black, draw=outlinecolor, fill=outlinecolor]
% Thin arrow to connect nodes
\tikzstyle{arrow_small} = [->, very thin, >=stealth, black]

\def\y{0.4}
\def\arrow{0.4}
\def\step{0.35}
\def\img{0.3}
\def\imgfinal{1.1}
\def\adc{0.22}

%-------------tikz neural network start-------------
\usepackage{amsmath} % for aligned
%\usepackage{amssymb} % for \mathbb
\usepackage{tikz}
\usepackage{bm}
%\usepackage{etoolbox} % for \ifthen
\usepackage{listofitems} % for \readlist to create arrays
\usetikzlibrary{arrows.meta} % for arrow size
\usepackage[outline]{contour} % glow around text
\contourlength{1.4pt}

\tikzset{>=latex} % for LaTeX arrow head
\usepackage{xcolor}
\colorlet{myred}{red!80!black}
\colorlet{myblue}{blue!80!black}
\colorlet{mygreen}{green!60!black}
\colorlet{myorange}{orange!70!red!60!black}
\colorlet{mydarkred}{red!30!black}
\colorlet{mydarkblue}{blue!40!black}
\colorlet{mydarkgreen}{green!30!black}
\tikzstyle{node}=[thick,circle,draw=myblue,minimum size=22,inner sep=0.5,outer sep=0.6]
\tikzstyle{node in}=[node,green!20!black,draw=mygreen!30!black,fill=mygreen!25]
\tikzstyle{node hidden}=[node,blue!20!black,draw=myblue!30!black,fill=myblue!20]
\tikzstyle{node convol}=[node,orange!20!black,draw=myorange!30!black,fill=myorange!20]
\tikzstyle{node out}=[node,red!20!black,draw=myred!30!black,fill=myred!20]
\tikzstyle{connect}=[thick,mydarkblue] %,line cap=round
\tikzstyle{connect arrow}=[-{Latex[length=4,width=3.5]},thick,mydarkblue,shorten <=0.5,shorten >=1]
\tikzset{ % node styles, numbered for easy mapping with \nstyle
  node 1/.style={node in},
  node 2/.style={node hidden},
  node 3/.style={node out},
}
\def\nstyle{int(\lay<\Nnodlen?min(2,\lay):3)} % map layer number onto 1, 2, or 3
